\documentclass[a4paper]{article}
\usepackage[english]{babel}
\usepackage[utf8]{vietnam}
\usepackage{minted}
\usepackage{csvsimple}
\usepackage{cases}
%\usepackage{vntex}

%\usepackage[english,vietnam]{babel}
%\usepackage[utf8]{inputenc}

%\usepackage[utf8]{inputenc}
%\usepackage[francais]{babel}
\usepackage{a4wide,amssymb,epsfig,latexsym,multicol,array,hhline,fancyhdr}

\usepackage{amsmath}
\usepackage{amsfonts}
\usepackage{amssymb}
\usepackage{lastpage}
\usepackage[lined,boxed,commentsnumbered]{algorithm2e}
\usepackage{enumerate}
\usepackage{color}
\usepackage{graphicx}
\usepackage{listings}

% Standard graphics package
\usepackage{array}
\usepackage{tabularx, caption}
\usepackage{multirow}
\usepackage{multicol}
\usepackage{rotating}
\usepackage{graphics}
\usepackage[a4paper,left=2cm,right=2cm,top=1.8cm,bottom=2.8cm]{geometry}
\usepackage{setspace}
\usepackage{epsfig}
\usepackage{tikz}
\usetikzlibrary{arrows,snakes,backgrounds}
\usepackage[unicode]{hyperref}
%can file puenc.def trong thu muc goc de option [unicode] tao ra bookmark bang tieng Viet
\hypersetup{urlcolor=blue,linkcolor=black,citecolor=black,colorlinks=true} 
%\usepackage{pstcol} 								
% PSTricks with the standard color package

\newtheorem{theorem}{{\bf Theorem}}
\newtheorem{property}{{\bf Property}}
\newtheorem{proposition}{{\bf Proposition}}
\newtheorem{corollary}[proposition]{{\bf Corollary}}
\newtheorem{lemma}[proposition]{{\bf Lemma}}


%\usepackage{fancyhdr}
\setlength{\headheight}{40pt}
\pagestyle{fancy}
\fancyhead{} % clear all header fields
\fancyhead[L]{
 \begin{tabular}{rl}
    \begin{picture}(25,15)(0,0)
    \put(0,-8){\includegraphics[width=8mm, height=8mm]{hcmut.png}}
    %\put(0,-8){\epsfig{width=10mm,figure=hcmut.eps}}
   \end{picture}&
	%\includegraphics[width=8mm, height=8mm]{hcmut.png} & %
	\begin{tabular}{l}
		\textbf{\bf \ttfamily HO CHI MINH CITY UNIVERSITY OF TECHNOLOGY, VNU-HCM}\\
		\textbf{\bf \ttfamily FACULTY OF COMPUTER SCIENCE \& ENGINEERNG}
	\end{tabular} 	
 \end{tabular}
}
\fancyhead[R]{
	\begin{tabular}{l}
		\tiny \bf \\
		\tiny \bf 
	\end{tabular}  }
\fancyfoot{} % clear all footer fields
\fancyfoot[L]{\scriptsize \ttfamily Assignment - Video Streaming with RTSP and RTP, 2020 - 2021}
\fancyfoot[R]{\scriptsize \ttfamily Page {\thepage}/\pageref{LastPage}}
\renewcommand{\headrulewidth}{0.3pt}
\renewcommand{\footrulewidth}{0.3pt}


%%%
\setcounter{secnumdepth}{4}
\setcounter{tocdepth}{3}
\makeatletter
\newcounter {subsubsubsection}[subsubsection]
\renewcommand\thesubsubsubsection{\thesubsubsection .\@alph\c@subsubsubsection}
\newcommand\subsubsubsection{\@startsection{subsubsubsection}{4}{\z@}%
                                     {-3.25ex\@plus -1ex \@minus -.2ex}%
                                     {1.5ex \@plus .2ex}%
                                     {\normalfont\normalsize\bfseries}}
\newcommand*\l@subsubsubsection{\@dottedtocline{3}{10.0em}{4.1em}}
\newcommand*{\subsubsubsectionmark}[1]{}
\makeatother

\begin{document}

\begin{titlepage}

\begin{center}
HO CHI MINH CITY UNIVERSITY OF TECHNOLOGY, VNU-HCM\\
FACULTY OF COMPUTER SCIENCE \& ENGINEERNG
\end{center}

\vspace{1cm}

\begin{figure}[H]
\begin{center}
\includegraphics[width=3cm]{hcmut.png}
\end{center}
\end{figure}

\vspace{1cm}


\begin{center}
\begin{tabular}{c}
\multicolumn{1}{l}{\textbf{{\Large COMPUTER NETWORK}}}\\
~~\\
\hline
\\
\multicolumn{1}{l}{\textbf{{\Large Assignment}}}\\
\\
\textbf{\Huge Video Streaming with RTSP and RTP}\\
\\
\hline
\end{tabular}
\end{center}

\vspace{3cm}

\begin{table}[h]
\begin{tabular}{rrl}

\hspace{5 cm} & Tutor: & Bùi Xuân Giang\\
& Class: & L05\\
& Student members: & Huỳnh Phạm Phước Linh - 1710165 \\
& & Trương Ngọc Trung Anh - 2020004 \\
& & Lê Văn Phong - 1712607 \\


\end{tabular}
\end{table}

\begin{center}
{\footnotesize Ho Chi Minh, 11/2020}
\end{center}
\end{titlepage}


%\thispagestyle{empty}
\selectlanguage{english}
\newpage
\tableofcontents
\newpage


%%%%%%%%%%%%%%%%%%%%%%%%%%%%%%%%%
\section{Introduction}

\indent Trong bài tập lớn 1, chúng sẽ triển khai một video streaming Server và Client giao tiếp bằng Real-Time Streaming Protocol (RTSP) và gửi dữ liệu bằng Real-time Transport Protocol (RTP). Nhiệm vụ của chúng ta là triển khai giao thức RTSP trong Client và triển khai tạo nhịp RTP trong Server. Chúng ta sẽ cung cấp mã triển khai giao thức RTSP trong Server, khử nhịp độ RTP trong Client và đảm nhận việc hiển thị video đã truyền.\\
\\
Bài tập lớn gồm 5 files: \\
\indent Client.py \\
\indent ClientLauncher.py \\
\indent RtpPacket.py \\
\indent Server.py \\
\indent ServerWorker.py 

%%%%%%%%%%%%%%%%%%%%%%%%%%%%%%%%%
\section{Running the code}

\subsection{Server.py}

Chạy Server.py: \\
\indent python Server.py server\_port \\
\indent or \\
\indent python3 Server.py server\_port \\

\noindent Trong đó: \\
\indent \textbf{server\_port} là cổng mà Server của bạn lắng nghe các kết nối RTSP đến.
\begin{enumerate}[\indent \indent \alph{enumi}.]
    \item Chúng ta có thể cho nó giá trị là 1025
    \item Cổng RTSP tiêu chuẩn là 554
    \item Giá trị được đặt cho server\_port phải lớn hơn 1024
\end{enumerate}

\subsection{ClientLauncher.py}

Chạy ClientLauncher.py: \\
\indent python ClientLauncher.py server\_host server\_port RTP\_port name\_video\_file \\
\indent or \\
\indent python3 ClientLauncher.py server\_host server\_port RTP\_port name\_video\_file \\

\noindent Trong đó: 
% \begin{enumerate}[\indent \arabic{enumi}.]
\begin{itemize}
    \item \textbf{server\_host} \tab là địa chỉ IP của máy cục bộ. (VD:"127.0.0.1")
    \item \textbf{server\_port} là cổng mà Server đang lắng nghe. (Vd: 1025)
    \item \textbf{RTP\_port} là cổng mà các gói RTP được nhận. (Vd: 8005)
    \item \textbf{name\_video\_file} là tên của video muốn xem (Vd: video.Mjpeg)
\end{itemize}

\subsection{Hướng dẫn sử dụng}
\noindent Sử dụng 2 terminal (1 - Server, 1 - Client):\\

Trong terminal thứ nhất, ta chạy dòng lệnh sau:
\begin{lstlisting}[language=Python3]
python Server.py 1025 
\end{lstlisting}

Trong terminal thứ hai, ta chạy dòng lệnh sau:
\begin{lstlisting}
python ClientLauncher.py 127.0.0.1 1025 8005 video.Mjpeg
\end{lstlisting}
\textbf{SETUP} để nhận video stream.\\
\textbf{PLAY} để phát video.\\
\textbf{PAUSE} để dừng video đang phát.\\
\textbf{TEARDOWN} để đóng Client terminal.

\section{Real-Time Streaming Protocol (RTSP)}
\begin{itemize}
\item Giao thức phát trực tuyến thời gian thực.\\
\item Kiểm soát các streaming media server đối với hệ thống giải trí và truyền thông.\\
\item Thiết lập và kiểm soát các media session giữa các điểm kết thúc.\\
\item Sử dụng TCP.\\
\end{itemize}


\section{Real-time Transport Protocol (RTP)}
\begin{itemize}
\item Giao thức vận tải thời gian thực.\\
\item Giao thức mạng để phân phối âm thanh và video qua IP Networks.\\
\item Thiết lập và kiểm soát các media session giữa các điểm kết thúc.\\
\item Sử dụng UDP.\\
\end{itemize}


%%%%%%%%%%%%%%%%%%%%%%%%%%%%%%%%%
\section{Python code}

%% Python code %%
\lstset{
language=Python,
frame=lines,
label={lst:code_direct},
basicstyle=\footnotesize,
breaklines=true
}
\begin{lstlisting}[language=Python]
class ServerWorker:
    SETUP = 'SETUP'
    PLAY = 'PLAY'
    PAUSE = 'PAUSE'
    TEARDOWN = 'TEARDOWN'
    
    INIT = 0
    READY = 1
    PLAYING = 2
    state = INIT
	
	OK_200 = 0
    FILE_NOT_FOUND_404 = 1
    CON_ERR_500 = 2
\end{lstlisting}
\\
%% End python code %%
\lstset{
language=Python,
frame=lines,
label={lst:code_direct},
basicstyle=\footnotesize,
breaklines=true
}
\begin{lstlisting}[language=Python]
class Client:
    INIT = 0
    READY = 1
    PLAYING = 2
    state = INIT
    
    SETUP = 0
    PLAY = 1
    PAUSE = 2
    TEARDOWN = 3
\end{lstlisting}
\\
\tab Client sẽ gửi đến Server thông qua Giao thức RTSP là các lệnh như:
\begin{itemize}
    \item SETUP
    \item PLAY
    \item PAUSE
    \item TEARDOWN
\end{itemize}
\\
\tab Các lệnh này sẽ cho phía Server biết hành động tiếp theo mà nó sẽ hoàn thành.
\tab Server sẽ trả lời Client thông qua Giao thức RTSP là các giá trị như:
\begin{itemize}
    \item OK\_200
    \item FILE\_NOT\_FOUND\_404
    \item CON\_ERR\_500
\end{itemize}
\\
\subsection{SETUP Command}
\tab Nếu lệnh SETUP được gửi từ Client đến Server.\\
%% Python code %%
\lstset{
language=Python,
frame=lines,
label={lst:code_direct},
basicstyle=\footnotesize,
breaklines=true
}
\begin{lstlisting}[language=Python]
def sendRtspRequest(self, requestCode):
    if requestCode == self.SETUP and self.state == self.INIT:
        threading.Thread(target=self.recvRtspReply).start()
        
        self.rtspSeq = 1
        
        request = "SETUP " + str(self.fileName) + "\n" + str(self.rtspSeq) + "\n" + " RTSP/1.0 RTP/UDP " + str(self.rtpPort)
        
        self.rtspSocket.send(request.encode('utf-8'))
        
        self.requestSent = self.SETUP
\end{lstlisting}
\\
%% End python code %%
Gói “SETUP” RTSP sẽ bao gồm:\\
    \begin{itemize}
        \item Lệnh SETUP
	    \item Tên video
	    \item RTSP Packet Sequence Number bắt đầu 1
	    \item Protocol type: RTSP/1.0 RTP
	    \item Transmission Protocol: UDP
	    \item Cổng RTP để truyền luồng video
    \end{itemize}
%% Python code %%
\lstset{
language=Python,
frame=lines,
label={lst:code_direct},
basicstyle=\footnotesize,
breaklines=true
}
\begin{lstlisting}[language=Python]
# Process SETUP request
    if requestType == self.SETUP:
        if self.state == self.INIT:
            # Update state
            print("processing SETUP\n")
			
            try:
                self.clientInfo['videoStream'] = VideoStream(filename)
                self.state = self.READY
            except IOError:
                self.replyRtsp(self.FILE_NOT_FOUND_404, seq[1])
            
            # Generate a randomized RTSP session ID
            self.clientInfo['session'] = randint(100000, 999999)
			
            # Send RTSP reply
            self.replyRtsp(self.OK_200, seq[0])
            
            # Get the RTP/UDP port from the last line
            self.clientInfo['rtpPort'] = request[2].split(' ')[3]

\end{lstlisting}
\\
%% End python code %%
Khi Server nhận được lệnh SETUP:
\begin{itemize}
    \item Gán cho Client một Specific Session Number ngẫu nhiên
    \item Nếu lệnh hoặc trạng thái Server lỗi, trả gói ERROR lại cho Client
    \item Nếu không có lỗi:\\
    \begin{lstlisting}[language=Python]
class VideoStream:
    def __init__(self, filename):
        self.filename = filename
        try:
    	    self.file = open(filename, 'rb')
        except:
    	    raise IOError
        self.frameNum = 0
    \end{lstlisting}
    \tab Server sẽ mở tệp video được chỉ định trong gói SETUP và khởi tạo số khung hình video của nó bằng 0.
    \tab Trả về OK\_200 cho Client và gán state = READY.
\end{itemize}
Client sẽ liên tục nhận Server's RTSP Reply
\lstset{
language=Python,
frame=lines,
label={lst:code_direct},
basicstyle=\footnotesize,
breaklines=true
}
\begin{lstlisting}[language=Python]
def recvRtspReply(self):
    while True:
        reply = self.rtspSocket.recv(1024)
        if reply:
            self.parseRtspReply(reply.decode("utf-8"))
        if self.requestSent == self.TEARDOWN:
            self.rtspSocket.shutdown(socket.SHUT_RDWR)
            self.rtspSocket.close()
            break
\end{lstlisting}
Sau đó phân tích cú pháp của gói RTSP Reply, nhận Session Number
\lstset{
language=Python,
frame=lines,
label={lst:code_direct},
basicstyle=\footnotesize,
breaklines=true
}
\begin{lstlisting}[language=Python]
def parseRtspReply(self, data):
    lines = data.split('\n')
    seqNum = int(lines[1].split(' ')[1])
    if seqNum == self.rtspSeq:
        session = int(lines[2].split(' ')[1])
        if self.sessionId == 0:
            self.sessionId = session
\end{lstlisting}
\\
Nếu như gói Reply được phản hồi là lệnh SETUP thì Client sẽ đặt giá trị state = READY.
\lstset{
language=Python,
frame=lines,
label={lst:code_direct},
basicstyle=\footnotesize,
breaklines=true
}
\begin{lstlisting}[language=Python]
# Process only if the session ID is the same
if self.sessionId == session:
    if int(lines[0].split(' ')[1]) == 200:
        if self.requestSent == self.SETUP:
            print ("Updating RTSP state...")
            self.state = self.READY
            print ("Setting Up RtpPort for Video Stream")
            self.openRtpPort() 
        elif self.requestSent == self.PLAY:
            self.state = self.PLAYING
            print ('-'*60 + "\nClient is PLAYING...\n" + '-'*60)
        elif self.requestSent == self.PAUSE:
            self.state = self.READY
            self.playEvent.set()
        elif self.requestSent == self.TEARDOWN:
            self.teardownAcked = 1 
\end{lstlisting}
Sau đó, mở cổng RTP để nhận video stream.
\lstset{
language=Python,
frame=lines,
label={lst:code_direct},
basicstyle=\footnotesize,
breaklines=true
}
\begin{lstlisting}[language=Python]
def openRtpPort(self):
    self.rtpSocket.settimeout(0.5)
        try:
            self.rtpSocket.bind((self.serverAddr,self.rtpPort))
            print ("Bind RtpPort Success")
        except:
            tkinter.messagebox.showwarning('Unable to Bind', 'Unable to bind PORT=%d' %self.rtpPort)
\end{lstlisting}

\subsection{PLAY Command}
Nếu lệnh PLAY được gửi từ Client đến Server
\lstset{
language=Python,
frame=lines,
label={lst:code_direct},
basicstyle=\footnotesize,
breaklines=true
}
\begin{lstlisting}[language=Python]
elif requestCode == self.PLAY and self.state == self.READY:
    self.rtspSeq = self.rtspSeq + 1
    request = "PLAY " + "\n" + str(self.rtspSeq)
    self.rtspSocket.send(request.encode("utf-8"))
    print ('-'*60 + "\nPLAY request sent to Server...\n" + '-'*60)
    self.requestSent = self.PLAY
\end{lstlisting}
Server sẽ tạo một Socket truyền từ RTP qua UDP và bắt đầu gửi gói video stream
\lstset{
language=Python,
frame=lines,
label={lst:code_direct},
basicstyle=\footnotesize,
breaklines=true
}
\begin{lstlisting}[language=Python]
elif requestType == self.PLAY:
    if self.state == self.READY:
    print("processing PLAY\n")
    self.state = self.PLAYING
    self.clientInfo["rtpSocket"] = socket.socket(socket.AF_INET, socket.SOCK_DGRAM)
    self.replyRtsp(self.OK_200, seq[0])
    self.clientInfo['event'] = threading.Event()
    self.clientInfo['worker']= threading.Thread(target=self.sendRtp)
    self.clientInfo['worker'].start()
\end{lstlisting}
VideoStream.py sẽ giúp cắt tệp video thành từng frame riêng biệt và đưa từng frame vào gói dữ liệu RTP
\lstset{
language=Python,
frame=lines,
label={lst:code_direct},
basicstyle=\footnotesize,
breaklines=true
}
\begin{lstlisting}[language=Python]
def nextFrame(self):
    data = self.file.read(5)
    if data: 
        framelength = int(data)
        data = self.file.read(framelength)
        self.frameNum += 1
    return data
\end{lstlisting}
Mỗi gói dữ liệu cũng sẽ được mã hóa với một header, header sẽ bao gồm: RTP-version filed,	Padding, extension, Contributing source, Marker,	Type Field,	Sequence Number, Timestamp, SSRC, Payload).\\
\begin{center}
\begin{table}[h!]
\centering
\begin{tabular}{ | c | c | c | c | c | c | c | c | }
 \hline
 \multicolumn{8}{|c|}{RTP Packet Header} \\
 \hline Bit Offset  & 0-1       & 2     & 3     & 4-7   & 8     & 9-15  & 16-31 \\
 \hline 0           & Version   & P     & X     & CC    & M     & PT    & Sequence Number\\
 \hline 32          & \multicolumn{7}{|c|}{Timestamp} \\
 \hline 64          & \multicolumn{7}{|c|}{SSRC identifier} \\
 \hline 
 \multirow{2}{2.5cm}{\centering 96}
                    & \multicolumn{7}{|c|}{CSRC identifiers} \\
                    & \multicolumn{7}{|c|}{...}  \\
 \hline 96+32xCC    & \multicolumn{6}{|c|}{Profile-specific extension header ID}    & Extension header length \\
 \hline 
  \multirow{2}{2.5cm}{\centering 128+32xCC}
                    & \multicolumn{7}{|c|}{Extension header} \\
                    & \multicolumn{7}{|c|}{...}  \\
 \hline 
\end{tabular}
\end{table}
\end{center}

\lstset{
language=Python,
frame=lines,
label={lst:code_direct},
basicstyle=\footnotesize,
breaklines=true
}
\begin{lstlisting}[language=Python]
class RtpPacket:
    header = bytearray(HEADER_SIZE)
    def __init__(self):
        pass
    def encode(self, version, padding, extension, cc, seqnum, marker, pt, ssrc, payload):
        timestamp = int(time())
        header = bytearray(HEADER_SIZE)
        header[0] = header[0] | version << 6 # 2 bit
        header[0] = header[0] | padding << 5 # 1 bit
        header[0] = header[0] | extension << 4 # 1 bit
        header[0] = header[0] | cc << 3 # 4 bit
        
        header[1] = header[1] | marker << 7 # 1 bit
        header[1] = header[1] | pt #  7 bit
        header[2] = (seqnum >> 8) & 0xFF # 16 bit, first 8
        header[3] = seqnum & 0xFF # second 8
        
        header[4] = (timestamp >> 24) & 0xFF # 32 bit timestamp
        header[5] = (timestamp >> 16) & 0xFF
        header[6] = (timestamp >> 8) & 0xFF
        header[7] = (timestamp >> 0) & 0xFF
        
        header[8] = (ssrc >> 24) & 0xFF # 32 bit ssrc
        header[9] = (ssrc >> 16) & 0xFF
        header[10] = (ssrc >> 8) & 0xFF
        header[11] = (ssrc >> 0) & 0xFF
        
        self.header = header
        self.payload = payload
\end{lstlisting}

\noindent Một RTP Packet sẽ bao gồm header và video frame sẽ được gửi đến RTP Port của Client.

\lstset{
language=Python,
frame=lines,
label={lst:code_direct},
basicstyle=\footnotesize,
breaklines=true
}
\begin{lstlisting}[language=Python]
def sendRtp(self):
    while True:
        self.clientInfo['event'].wait(0.05) 
        if self.clientInfo['event'].isSet(): 
            break 
        data = self.clientInfo['videoStream'].nextFrame()
        if data: 
            frameNumber = self.clientInfo['videoStream'].frameNbr()
            try:
                address = self.clientInfo['rtspSocket'][1][0]
                port = int(self.clientInfo['rtpPort'])
                #================= Here =================#
                self.clientInfo['rtpSocket'].sendto(self.makeRtp(data, frameNumber),(address,port))
            except:
                print("Connection Error")
\end{lstlisting}

Client sẽ decode RTP Packet để lấy header và video frame, tổ chức lại các frame và hiển thị trên giao diện người dùng (User Interface).

\lstset{
language=Python,
frame=lines,
label={lst:code_direct},
basicstyle=\footnotesize,
breaklines=true
}
\begin{lstlisting}[language=Python]
def decode(self, byteStream):
    self.header = bytearray(byteStream[:HEADER_SIZE])
    self.payload = byteStream[HEADER_SIZE:]
\end{lstlisting}

\subsection{PAUSE Command}

\noindent Nếu lệnh PAUSE được gửi từ Client đến Server, nó sẽ dừng gửi các video frame từ Server đến Client.

\lstset{
language=Python,
frame=lines,
label={lst:code_direct},
basicstyle=\footnotesize,
breaklines=true
}
\begin{lstlisting}[language=Python]
elif requestCode == self.PAUSE and self.state == self.PLAYING:
    self.rtspSeq = self.rtspSeq + 1
    request = "PAUSE " + "\n" + str(self.rtspSeq)
    self.rtspSocket.send(request.encode("utf-8"))
    print ('-'*60 + "\nPAUSE request sent to Server...\n" + '-'*60)
    self.requestSent = self.PAUSE
\end{lstlisting}
\\

\noindent Nếu như gói Reply nhận lệnh PAUSE thì Client sẽ đặt giá trị state = READY.

\lstset{
language=Python,
frame=lines,
label={lst:code_direct},
basicstyle=\footnotesize,
breaklines=true
}
\begin{lstlisting}[language=Python]
elif self.requestSent == self.PAUSE:
    self.state = self.READY
\end{lstlisting}
\\

\subsection{TEARDOWN Command}

Nếu lệnh TEARDOWN được gửi từ Client đến Server, nó sẽ ngăn Server gửi các video frame đến Client và đóng tất cả Client terminal.

\lstset{
language=Python,
frame=lines,
label={lst:code_direct},
basicstyle=\footnotesize,
breaklines=true
}
\begin{lstlisting}[language=Python]
elif requestCode == self.TEARDOWN and not self.state == self.INIT:
    self.rtspSeq = self.rtspSeq + 1
    request = "TEARDOWN " + "\n" + str(self.rtspSeq)
    self.rtspSocket.send(request.encode("utf-8"))
    print ('-'*60 + "\nTEARDOWN request sent to Server...\n" + '-'*60)
    self.requestSent = self.TEARDOWN
\end{lstlisting}
\\

\lstset{
language=Python,
frame=lines,
label={lst:code_direct},
basicstyle=\footnotesize,
breaklines=true
}
\begin{lstlisting}[language=Python]
elif self.requestSent == self.TEARDOWN:
    self.teardownAcked = 1 
\end{lstlisting}
\\

%%%%%%%%%%%%%%%%%%%%%%%%%%%%%%%%% References
\begin{thebibliography}{80}


\bibitem{http://wikipedia:2020} wikipedia.
``\textbf{link: http://en.wikipedia.org/}'',
\textit{}, last access: 15/11/2020.

\end{thebibliography}
\end{document}

